\documentclass[twoside,a4paper,11pt,halfparskip,DIV=11,notitlepage]{scrartcl}
\usepackage{scrpage2}
\usepackage{amsmath}
\usepackage[pdfauthor={},pdftitle={},pdfstartview=FitH,pdfborder={0 0 0}]{hyperref}
\usepackage[ngerman]{babel}
\usepackage{float}
\usepackage{tikz}
\usetikzlibrary{positioning,arrows}
\usepackage[european]{circuitikz}
\usepackage{pgfplots}
\pgfplotsset{compat=1.16}
\usepgfplotslibrary{smithchart}

\usepackage{siunitx}
\usepackage{booktabs}
\usepackage{pifont}
\usepackage{xspace}
\DeclareUnicodeCharacter{2460}{\ding{172}\xspace}
\DeclareUnicodeCharacter{2461}{\ding{173}\xspace}
\renewcommand{\textdegree}{\si{\degree}\xspace}

\definecolor{HB9UFblue}{RGB}{0,61,165}
\definecolor{HB9UFred}{HTML}{ED135A}

\newcommand{\uline}[1]{%
    \tikz[baseline=(todotted.base)]{
        \node[inner sep=1pt,outer sep=0pt] (todotted) {#1};
        \draw[color=HB9UFblue,thick] (todotted.south west) -- (todotted.south east);
    }%
}%

\newcommand{\udash}[1]{%
    \tikz[baseline=(todotted.base)]{
        \node[inner sep=1pt,outer sep=0pt] (todotted) {#1};
        \draw[dashed,color=HB9UFred,thick] (todotted.south west) -- (todotted.south east);
    }%
}%

\newcommand{\Ohm}{$\Omega$\xspace}

\newcommand*\cleartoleftpage{%
  \clearpage
  \ifodd\value{page}\hbox{}\newpage\fi
}

\title{NanoVNA Workshop}
\author{UHF-Gruppe der USKA, Version 0.1}
\date{Mathias Weyland, HB9FRV} % FIXME: Und andere
\pagestyle{scrheadings}
%\setkomafont{pagehead}{\sf}

% FIXME
\usepackage{tikz}
\usetikzlibrary{shadows}
\addtokomafont{descriptionlabel}{\normalfont}
\usepackage{tcolorbox}
\tcbuselibrary{skins,breakable}

\tcbset{
mybox/.style={
  breakable,
  enhanced standard,
  boxrule=0.4pt,titlerule=-0.2pt,drop fuzzy shadow,
  width=\linewidth,
  title style={top color=myblue!30,bottom color=myblue!0.5},
  overlay unbroken and first={
    \path[fill=myblue]
    ([xshift=5pt,yshift=-\pgflinewidth]frame.north west) to[out=0,in=180] ([xshift=20pt,yshift=-5pt]title.south west) -- ([xshift=-20pt,yshift=-5pt]title.south east) to[out=0,in=180] ([xshift=-5pt,yshift=-\pgflinewidth]frame.north east) -- cycle;
  },
  fonttitle=\Large\bfseries\sffamily,
  fontupper=\sffamily,
  fontlower=\sffamily,
  before=\par\medskip\noindent,
  after=\par\medskip,
  center title,
  toptitle=3pt,
  top=11pt,topsep at break=-5pt,
  colback=white
}}

\newtcolorbox{lernziele}[1][\linewidth]{mybox,width=#1,title=Lernziele}

\newcounter{uebungscounter}
\newcommand{\uebung}[1]{
    \stepcounter{uebungscounter}
    \section{Übung \arabic{uebungscounter}: #1}
    \renewcommand{\labelenumi}{\arabic{uebungscounter}.\arabic{enumi}}
}
\newcommand{\loesungen}{
    \section*{Lösungen zu Übung \arabic{uebungscounter}}
}
%\setcounter{secnumdepth}{0} % sections are level 1

\sloppy

\begin{document}
\maketitle
\thispagestyle{empty}

\setcounter{tocdepth}{1}
\tableofcontents

\vfill

%\includegraphics[width=\textwidth]{titelbild.png} % FIXME

\vfill

\newpage

%   \uebung{Erste Schritte}
%   In dieser Übung trainieren Sie den Umgang mit dem GNU Radio Companion (GRC) und lernen erste Blöcke kennen.

%   \begin{lernziele}
%   \begin{itemize}
%   \item Sie können in GNU Radio Blöcke einfügen, löschen, verbinden und trennen.
%   \item Sie können einfache Wellenformen mit \hl{Signal Source} generieren.
%   \item Sie können Signale im Zeit- und Frequenzraum visualisieren.
%   \item Sie wissen, dass das Spektrum eigentlich auch negative Frequenzen aufweist.
%   \item Sie können Flows in GRC starten und stoppen.
%   \item Sie verstehen, wann der \hl{Throttle}-Block zum Einsatz kommt.
%   \end{itemize}
%   \end{lernziele}

% FIXME: Einleitung

\section{Grundlagen}

\subsection{Ports, Reflexions- und Transmissionsmessung}
Netzwerkanalysatoren sind Geräte, mit denen Transmissions- und
Reflexionseigenschaften von Hochfrequenznetzwerken frequenzabhängig ermittelt
werden können. Sie dienen dem Ausmessen von Antennen, Filtern, Transformatoren,
Baluns, Matching-Netzwerken, Splittern, Verstärkern, Stubs und vielem mehr. Ein
vollwertiger Netzwerkanalysator hat zwei Anschlüsse (sogenannte Ports), ① und ②.
Das Gerät kann sowohl an Port ① als auch an Port ② ein Prüfsignal (einen 
Stimulus) generieren. Bei aktiviertem Stimulus wird gemessen, wie viel des
Signals reflektiert wird (also in den aussendenden Port zurück gelangt), und
wie viel davon das zu testende Netzwerk durchwandert und zum jeweils anderen
Port gelangt (Transmission).

Viele Geräte -- darunter auch der NanoVNA -- implementieren vielfach nur einen Teil
dieser Messungen. Manchmal kommt der Stimulus nur aus einem der beiden Ports
und der Prüfling muss von Hand umgedreht werden; oder es sind nur
Reflexionsmessungen möglich, letzteres z.B. bei einem Antennen-Analyzer. So
oder so wird entweder nur die Amplitude (skalar) oder die Amplitude und Phase
(Vektor) gemessen (siehe Abschnitt \ref{sec:vectors}).

Die Reflexions- und die Transmissionsmessung sind in
Abbildung \ref{fig:testset} schematisch dargestellt: Ein aus Port ① ausgehender
Stimulus trifft auf einen Prüfling (ein Filter, eine Antenne etc.). Ein Teil
der eintreffenden Energie wird zurück zu Port ① reflektiert und dort gemessen
(Reflexionsmessung). Der Rest wird entweder vom Prüfling in Wärme umgewandelt,
als Radiowelle abgestrahlt oder gelangt (falls vorhanden) an den Ausgang des
Prüflings. Die Energie am Ausgang des Prüflings wird Port ② zugeführt
und dort gemessen (Transmissionsmessung).

\begin{figure}[H]
    \begin{center}
        \begin{tikzpicture}[text width=6em, text centered,, node distance=8em]
            \tikzstyle{box}=[draw,rounded corners=3pt,minimum height=3em]
            \node[box] (p1) {\sf\Large{Port ①}};
            \node[box,right=of p1] (dut) {\sf\Large{DUT}};
            \node[box,right=of dut] (p2) {\sf\Large{Port ②}};

            \draw[-stealth',thick,transform canvas={yshift=2mm}] (p1) -- node[above] {\sf Stimulus} (dut);
            \draw[-stealth',thick,transform canvas={yshift=-2mm}] (dut) -- node[below]{\sf Reflexion} (p1);
            \draw[-stealth',thick] (dut) -- node[above] {\sf Transmission} (p2);
        \end{tikzpicture}
    \end{center}
    \caption{Ein aus Port ① ausgehender Stimulus trifft auf einen Prüfling (device under test (DUT), z.B
        ein Filter, eine Antenne etc.). Ein Teil der eintreffenden Energie wird zurück zu Port ① reflektiert
        (Reflexionsmessung). Der Rest wird entweder vom Prüfling in Wärme umgewandelt, als Radiowelle
        abgestrahlt oder gelangt (falls vorhanden) an den Ausgang des Prüflings. Die Energie am Ausgang
        des Prüflings wird Port ② zugeführt und dort gemessen (Transmissionsmessung).}
    \label{fig:testset}
\end{figure}

\subsection{Rückführungsdämpfung, Einfügedämpfung und Standing Wave Ratio}

Die durch Reflexion zurückgeführte Leistung wird mit der Leistung des Stimulus verglichen. Reflektiert
der Prüfling die gesamte Leistung, so erhält Port ① genausoviel Leistung zurück, wie er im Stimulus
ausgesandt hat (Kabeldämpfungen werden durch Kalibrierung korrigiert und sind deshalb hier vernachlässigt,
siehe Abschnitt \ref{sec:calibration}). Da die beiden Leistungen gleich gross sind, spricht man von
einer Rückführungsdämpfung von 0~dB. Reflektiert der Prüfling die halbe Leistung des Stimulus so spricht
man von einer Rückführungsdämpfung von 3~dB. Oft wird die Rückführungsdämpfung negativ angegeben, also
in diesem Fall -3~dB, was nicht ganz korrekt ist - es handelt sich ja schliesslich um eine Dämpfung 
und -3~dB Dämpfung wären +3~dB Verstärkung, was in dieser passiven Situation nicht eintreten kann.
Im Folgenden wird diese Konvention trotzdem verwendet, weil sie sich eingebürgert hat.

% FIXME: Einfügedämpfung: Definieren, sicherstellen dass diese Grösse richtig ist.

Das im Amateurfunk beliebte Stehwellenverhältnis (standing wave ratio, SWR) bezeichnet eigentlich
ein Amplitudenverhältnis von speziellen Stellen entlang einer Leitung (z.B. Koaxialkabel oder
Hühnerleiter). Bei der Bestimmung des SWR wird aber in den seltensten Fällen die Leitung an sich
untersucht. Oft ist die Leitung auch derart kurz, dass diese speziellen Stellen gar nicht auftreten.
Der Begriff ist deshalb leider etwas fehlgeleitet. Tatsächlich wird in aller Regel die Rückführungsdämpfung
gemessen und mit den entsprechenden Formeln in ein Stehwellenverhältnis umgesetzt. Tabelle \ref{tab:rlswr}
gibt den Zusammenhang zwischen Rückführungsdämpfung und SWR für ausgewählte Grössen an.

\begin{table}[H]
    \caption{Gegenübertellung von SWR, Rückführungsdämpfung und dem prozentualen Anteil reflektierter Leistung.}
    \label{tab:rlswr}
\begin{center}\begin{tabular}{SSS}\toprule
{\textbf{SWR}} & {\textbf{Rückführung}} & {\textbf{\% Reflkt.}}\\\midrule
1.0 & {-$\infty$} & 0.0\\
1.1 & -26.44 & 0.2\\
1.2 & -20.83 & 0.8\\
1.3 & -17.69 & 1.7\\
1.4 & -15.56 & 2.8\\
1.5 & -13.98 & 4.0\\
1.6 & -12.74 & 5.3\\
1.7 & -11.73 & 6.7\\
1.8 & -10.88 & 8.2\\
1.9 & -10.16 & 9.6\\
2.0 & -9.54 & 11.1\\
2.2 & -8.52 & 14.1\\
2.4 & -7.71 & 17.0\\
2.6 & -7.04 & 19.8\\
2.8 & -6.49 & 22.4\\\bottomrule
\end{tabular}
\hspace{1cm}
\begin{tabular}{SSS}\toprule
{\textbf{SWR}} & {\textbf{Rückführung}} & {\textbf{\% Reflkt.}}\\\midrule
3 & -6.02 & 25.0\\
4 & -4.44 & 36.0\\
5 & -3.52 & 44.4\\
6 & -2.92 & 51.0\\
7 & -2.5 & 56.3\\
8 & -2.18 & 60.5\\
9 & -1.94 & 64.0\\
10 & -1.74 & 66.9\\
20 & -0.87 & 81.9\\
30 & -0.58 & 87.5\\
40 & -0.43 & 90.6\\
50 & -0.35 & 92.3\\
60 & -0.29 & 93.5\\
80 & -0.22 & 95.1\\
100& -0.17 & 96.2\\\bottomrule
\end{tabular}\end{center}
\end{table}

Eine andere Darstellung des selben Zusammenhangs ist im Nomogramm in Abbildung
\ref{fig:rlnomogramm} zu sehen.

\begin{figure}[t]
    % FIXME: Check if the math is right
    \begin{center}\begin{tikzpicture}[scale=1.25]
        % Return loss
        % [13, 10, 8.5, 7.5, 6.5, 5.5, 4.5, 3.5, 2.5, 1.5, 0.9, 0.7, 0.5, 0.3, 0.1]
        % [9, 8, 7, 6, 5, 4, 3, 2, 1, 0.8, 0.6, 0.4, 0.2, 0]
        % for i in y1: print("%.1f / %f, " % (i,  10**(-i/10)),)
        \foreach \rl/\x in {13 / 0.050119, 10 / 0.1, 8.5 / 0.141254, 7.5 / 0.177828, 6.5 / 0.223872, 5.5 / 0.281838, 4.5 / 0.354813, 3.5 / 0.446684, 2.5 / 0.562341, 1.5 / 0.707946, 0.9 / 0.812831, 0.7 / 0.851138, 0.5 / 0.891251, 0.3 / 0.933254, 0.1 / 0.97723}
        {
            \node[above] at ({\x*10},0.) {\sf\tiny{\rl}};
            \draw ({\x*10},0.05) -- ({\x*10},-0.05);
        }
        \foreach \rl/\x in { 9 / 0.125893, 8 / 0.158489, 7 / 0.199526, 6 / 0.251189, 5 / 0.316228, 4 / 0.398107, 3 / 0.501187, 2 / 0.630957, 1 / 0.794328, 0.8 / 0.831764, 0.6 / 0.870964, 0.4 / 0.912011, 0.2 / 0.954993, 0 / 1.000000}
        {
            \node[below] at ({\x*10},0.) {\sf\tiny{\rl}};
            \draw ({\x*10},0.05) -- ({\x*10},-0.05);
        }
        \node[above] at (0,0) {\sf\tiny{$\infty$}};
        \draw (0,0.05) -- (0,-0.05);
        \draw (0,0) -- (10,0);
        \node[right] at (10,0) {~\sf\scriptsize{RL (dB)}};

        % SWR
        % FIXME: infty
        % [1, 1.5, 1.9, 2.5, 3.5, 4.5, 5.5, 6.5, 7.5, 8.5, 9.5, 12, 16, 20, 40]
        % [1.3, 1.7, 2, 3, 4, 5, 6, 7, 8, 9, 10, 14, 18, 30, 50]
        %for i in x: print("%.1f / %f, " % (i,  ((i-1) / (i+1))**2 ))
        \begin{scope}[yshift=-20]
        \foreach \s/\x in {1.0 / 0.000000, 1.5 / 0.040000, 1.9 / 0.096314, 2.5 / 0.183673, 3.5 / 0.308642, 4.5 / 0.404959, 5.5 / 0.479290, 6.5 / 0.537778, 7.5 / 0.584775, 8.5 / 0.623269, 9.5 / 0.655329, 12 / 0.715976, 16 / 0.778547, 20 / 0.818594, 40 / 0.904819}
        {
            \node[above] at ({\x*10},0.) {\sf\tiny{\s}};
            \draw ({\x*10},0.05) -- ({\x*10},-0.05);
        }
        \foreach \s/\x in {1.3 / 0.017013, 1.7 / 0.067215, 2 / 0.111111, 3 / 0.250000, 4 / 0.360000, 5 / 0.444444, 6 / 0.510204, 7 / 0.562500, 8 / 0.604938, 9 / 0.640000, 10 / 0.669421, 14 / 0.751111, 18 / 0.800554, 30 / 0.875130, 50 / 0.923106 }
        {
            \node[below] at ({\x*10},0.) {\sf\tiny{\s}};
            \draw ({\x*10},0.05) -- ({\x*10},-0.05);
        }
        \node[above] at (10,0) {\sf\tiny{$\infty$}};
        \draw (10,0.05) -- (10,-0.05);
        \draw (0,0) -- (10,0);
        \node[right] at (10,0) {~\sf\scriptsize{SWR}};
        \end{scope}

        % Reflected power [%]
        \begin{scope}[yshift=-40]
         \foreach \x in {0,10,20,...,100}
         {
             \node[above] at ({\x/10},0.) {\sf\tiny{\x}};
             \draw (\x/10,0.05) -- (\x/10,-0.05);
         }
        \draw (0,0) -- (10,0);
        \node[right] at (10,0) {~\sf\scriptsize{\% Reflkt.}};
        \end{scope}
    \end{tikzpicture}\end{center}
    \caption{Nomogramm des Zusammenhangs zwischen Rückfürungsdämpfung (return loss, RL),
    Stehwellenverhältnis (standing wave ratio, SWR) und dem Anteil der reflektierten Leistung (\% Reflkt.).}
    \label{fig:rlnomogramm}
\end{figure}


Die Einfügedämpfung ist ebenfalls ein Leistungsverhältnis: Die Leistung, die
bei eingesetztem Prüfling zu Port ② gelangt wird verglichen mit der Leistung
in Port ②, wenn der Prüfling nicht eingesetzt ist. Dieser Sachverhalt ist in
Abbildung \ref{fig:insertionloss} dargestellt. Die Einfügedämpfung ist dabei
definiert als $d=P_b/P_a$ bzw $10\cdot\log{(d)}$ in Dezibel. Daraus wird erkennbar,
dass ein ``perfekt transparenter'' Prüfling eine Einfügedämpfung von 0~dB
hat. Sorgt ein Prüfling dafür, dass nur noch die halbe Leistung zu Port ②
gelangt, so hat er eine Einfügedämpfung von 3~dB usw. Auch hier wird manchmal
die Einfügedämpfung salopperweise negativ angegeben, was eigentlich falsch ist.

\begin{figure}[H]
    \begin{center}
        \begin{tikzpicture}[text width=5em,text centered,node distance=4em]
            \tikzstyle{box}=[draw,rounded corners=3pt,minimum height=2em]
            \node[box] (p1) {\sf{Port ①}};
            \node[box] (p2) at (10,0) {\sf{Port ②}, $P_a$};

            \draw[-stealth',thick] (p1) --  (p2);
        \end{tikzpicture}

        \vspace{5mm}

        \begin{tikzpicture}[text width=5em,text centered,node distance=4em]
            \tikzstyle{box}=[draw,rounded corners=3pt,minimum height=2em]
            \node[box] (p1) {\sf{Port ①}};
            \node[box] (dut) at (5,0) {\sf{DUT}};
            \node[box] (p2) at (10,0) {\sf{Port ②}, $P_b$};

            \draw[-stealth',thick] (p1) --  (dut);
            \draw[-stealth',thick] (dut) -- (p2);
        \end{tikzpicture}
    \end{center}
    \caption{Erklärung Einfügedämpfung: Die Leistung $P_a$ ohne Prüfling wird
    verglichen zur Leistung $P_b$ mit Prüfling (device under test, DUT).}
    \label{fig:insertionloss}
\end{figure}

\subsection{Reaktive Lasten}\label{sec:vectors}
Bis jetzt wurden sämtliche Überlegungen anhand von Verhältnissen zwischen
aus- und eingehenden Leistungen gemacht. Messungen dieser Grössen werden
skalare Messungen bezeichnet. Der NanoVNA bietet allerdings die Möglichkeit,
nicht nur skalar, sondern vektoriell zu arbeiten. Das bedeutet, dass er auch
Informationen über die Phase der ein- und ausgehenden Signale verarbeiten kann.

Dies ist im Zusammenhang mit reaktiven Elementen (Spule/Induktivität,
Kondensator/Kapazität, Antenne ausserhalb der Resonanz etc.) wichtig: Reaktive
Elemente haben einen Blindwiderstand, der in aller Regel frequenzabhängig ist.
Dies ist in Abbildung \ref{fig:XRLC} dargestellt: Der Blindwiderstand $X$ eines
Widerstandes $R$ ist $X_R=0~\Omega$ ungeachtet der Frequenz, der Blindwiderstand
einer (idealen) Induktivität $L$ nimmt mit der Frequenz $f$ zu ($X_L=2\pi f L$)
und der Blindwiderstand einer Kapazität $C$ nimmt mit zunehmender Frequenz $f$
ab ($X_C=(2\pi f C)^{-1}$).

Spulen und Kondensatoren sind in Bezug auf ihren Blindwiderstand duale Elemente:
Werden sie in Serie geschaltet, so wirken $X_L$ und $X_C$ entgegengesetzt, und
wenn beide gleich gross sind, dann heben sie sich genau auf. In einem
Schwingkreis oder einer Antenne ist dies genau bei Resonanz erfüllt (vgl. auch
Abbildung \ref{fig:resonanzplot} in Abschnitt \ref{sec:schwingkreis}). Weil
$X_L$ und $X_C$ entgegengesetzt wirken, wird der kapazitive Blindwiderstand
$X_C$ oft negativ angegeben. So wird erkennbar, ob der Blindwiderstand induktiv
oder kapazitiv ist. Auch über diese Konvention lässt sich debattieren.

\begin{figure}
    \begin{tikzpicture}
        \begin{axis}[
            title = {Widerstand},
            xmin=0, xmax=10,
            ymin=-100,ymax=500,
            grid=both,
            xlabel={Frequenz $f$/Hz},
            ylabel={\uline{Blindwiderstand $X/\Omega$}},
            xticklabels={},
            yticklabels={},
            ytick = {-400,-200,0,200,400},
            height=4cm,
            width=0.3\textwidth,
            major grid style={black!10}
        ]
        \addplot [mark=none,samples=100,domain=0:10,color=HB9UFblue,thick] {0};
        \end{axis}
        \begin{axis}[
            xmin=0, xmax=10,
            ymin=-100,ymax=500,
            grid=none,
            separate axis lines,
            ylabel style = {align=center},
            ticks=none,
            xmajorticks=true,
            xminorticks=false,
            ymajorticks=true,
            yminorticks=false,
            scaled ticks=false,
            xtick = {0},
            ytick = {0},
            axis y line*=right,
            height=4cm,
            width=0.3\textwidth,
            major grid style={black!10}
        ]
        \addplot [mark=none,samples=100,domain=0:10,color=HB9UFred,dashed,thick] {300};
        \end{axis}
    \end{tikzpicture}\hfill
    \begin{tikzpicture}
        \begin{axis}[
            title = {Induktivität},
            xmin=0, xmax=10,
            ymin=-100,ymax=500,
            grid=both,
            xlabel={Frequenz $f$/Hz},
            xticklabels={},
            yticklabels={},
            ytick = {-400,-200,0,200,400},
            height=4cm,
            width=0.3\textwidth,
            major grid style={black!10}
        ]
        \addplot [mark=none,samples=100,domain=0:10,color=HB9UFblue,thick] {40*x};
        \end{axis}
        \begin{axis}[
            xmin=0, xmax=10,
            ymin=-100,ymax=500,
            grid=none,
            separate axis lines,
            ylabel style = {align=center},
            ticks=none,
            xmajorticks=true,
            xminorticks=false,
            ymajorticks=true,
            yminorticks=false,
            scaled ticks=false,
            xtick = {0},
            ytick = {0},
            axis y line*=right,
            height=4cm,
            width=0.3\textwidth,
            major grid style={black!10}
        ]
        \addplot [mark=none,samples=100,domain=0:10,color=HB9UFred,dashed,thick] {0};
        \end{axis}
    \end{tikzpicture}\hfill
    \begin{tikzpicture}
        \begin{axis}[
            title = {Kapazität},
            xmin=0, xmax=10,
            ymin=-100,ymax=500,
            grid=both,
            xlabel={Frequenz $f$/Hz},
            xticklabels={},
            yticklabels={},
            ytick = {0,200,400},
            height=4cm,
            width=0.3\textwidth,
            major grid style={black!10}
        ]
        \addplot [mark=none,samples=100,domain=0.2:10,color=HB9UFblue,thick] {200/x};
        \end{axis}
        \begin{axis}[
            xmin=0, xmax=10,
            ymin=-100,ymax=500,
            grid=none,
            separate axis lines,
            ylabel style = {align=center},
            ylabel={\udash{Wirkwiderstand $R/\Omega$}},
            ticks=none,
            xmajorticks=true,
            xminorticks=false,
            ymajorticks=true,
            yminorticks=false,
            scaled ticks=false,
            xtick = {0},
            ytick = {0},
            axis y line*=right,
            height=4cm,
            width=0.3\textwidth,
            major grid style={black!10}
        ]
        \addplot [mark=none,samples=100,domain=0:10,color=HB9UFred,dashed,thick] {0};
        \end{axis}
    \end{tikzpicture}
    \caption{Blindwiderstand eines Widerstandes (Bauteil), einer Induktivität ($X_L$) und einer
    Kapazität ($X_C$) von links nach rechts. Der Wirkwiderstand ist jeweils gestrichelt gezeichnet.}
    \label{fig:XRLC}
\end{figure}

\subsection{Phasenverschiebung mit reaktiven Lasten}
Ein Spannungsteiler bestehend aus zwei 100~$\Omega$ (Wirk-)Widerständen halbiert die Eingangspannung,
bei Gleichspannung wie auch bei Wechselspannung. % FIXME SIehe Abbildung ...?
Es wäre nun naheliegend anzunehmen, dass man einen dieser beiden Widerstände
durch eine Induktivität oder Kapazität mit einem Blindwiderstand von 100~$\Omega$
ersetzen könnte und nach wie vor über einen Spannungsteiler verfügt, welcher
die Eingangsspannung halbiert, das ist allerdings nicht der Fall, wie Abbildung
\ref{fig:phasenverschiebung} aufzeigt: Oben erfolgt die Halbierung der Spannung
wie beschrieben. In der unteren Hälfte der Abbildung ist der 100~$\Omega$
Widerstand durch eine Spule ersetzt, welche bei der Testfrequenz von $f=$100~MHz
$X_L=100$~$\Omega$ aufweist. Dennoch weist $U_L$ mehr als die halbe Amplitude von
$U_Q$ auf; ausserdem wird die Phasenverschiebung sichtbar. Diese Beobachtungen
lassen eine Reihe von Schlussfolgerungen zu, die im Weiteren wichtig sein werden:

\begin{itemize}
    \item Die Blindwiderstände der betrachteten Komponenten sind frequenzabhängig,
        Wirkwiderstände sind es nicht.
    \item Blindwiderstände führen zu Phasenverschiebungen, Wirkwiderstände nicht.
    \item Wirk- und Blindwiderstände dürfen nicht ``einfach so'' miteinander
        verrechnet werden, denn das so gefundene Resultat mit dem Spannungsteiler 
        stimmt nicht mit dem Experiment überein. Wenn, dann gibt es strikte und
        offenbar nicht-triviale Regeln, wie gerechnet werden muss.
\end{itemize}

Übrigens hat eine Spule in der Praxis zusätzlich zu Blindwiderstand auch immer
selbst einen Wirkwiderstand, weil der Draht, aus dem sie gewickelt ist,
verlustbehaftet ist. 


\begin{figure}
    \scalebox{0.8}{\begin{circuitikz}
        \draw (0,0) to[sinusoidal voltage source=$U_Q$] (0,-4)
              (0,0) to[short] (3,0) to[R,a=100~$\Omega$] (3,-2) to[short,-o] (5,-2)
              (3,-2) to[R,a=100~$\Omega$,*-*] (3,-4) to[short,-o] (5,-4)
              (3,-4) to[short,-*] (0,-4) node[ground] {};
        \draw[-stealth',shorten >=3mm,shorten <=3mm] (5,-2) -- node[right]{$U_L$} (5,-4);
     \end{circuitikz}}
    \hfill
    \begin{tikzpicture}
        \begin{axis}[
            xmin=0, xmax=10,
            ymin=-1.25,ymax=1.25,
            ylabel={Spannung / V},
            grid=both,
            xlabel={Zeit / ns},
            yticklabels={},
            xtick = {0,2.5,5,7.5,10},
            ytick = {-1,-0.5,0,0.5,1},
            height=4cm,
            width=0.6\textwidth,
            major grid style={black!10}
        ]
        \addplot [mark=none,samples=100,domain=0:10,color=HB9UFblue,thick] {0.5*sin(3600*x)} node[below,pos=0.45] {$U_L$};
        \addplot [mark=none,samples=100,domain=0:10,color=HB9UFred,dashed,thick] {sin(3600*x)} node[above,pos=0.45] {$U_Q$};;
        \end{axis}
    \end{tikzpicture}

    \vspace{5mm}

    \scalebox{0.8}{\begin{circuitikz}
        \draw (0,0) to[sinusoidal voltage source=$U_Q$] (0,-4)
              (0,0) to[short] (3,0) to[R,a=100~$\Omega$] (3,-2) to[short,-o] (5,-2)
              (3,-2) to[L,a=160~nH,*-*] (3,-4) to[short,-o] (5,-4)
              (3,-4) to[short,-*] (0,-4) node[ground] {};
        \draw[-stealth',shorten >=3mm,shorten <=3mm] (5,-2) -- node[right]{$U_L$} (5,-4);
     \end{circuitikz}}
    \hfill
    \begin{tikzpicture}
        \begin{axis}[
            xmin=0, xmax=10,
            ymin=-1.25,ymax=1.25,
            ylabel={Spannung / V},
            grid=both,
            xlabel={Zeit / ns},
            yticklabels={},
            xtick = {0,2.5,5,7.5,10},
            ytick = {-1,-0.5,0,0.5,1},
            height=4cm,
            width=0.6\textwidth,
            major grid style={black!10}
        ]
        \addplot [mark=none,samples=100,domain=0:10,color=HB9UFblue,thick] {0.7*sin(3600*x+45)} node[below,pos=0.45] {$U_L$};
        \addplot [mark=none,samples=100,domain=0:10,color=HB9UFred,dashed,thick] {sin(3600*x)} node[above,pos=0.45] {$U_Q$};;
        \end{axis}
    \end{tikzpicture}
    \caption{Oben: Ein Spannungsteiler halbiert die Spannung: $U_L$ hat die
    halbe Amplitude von $U_Q$. Unten: Phasenverschiebung durch reaktives Element. Bei
einer Frequenz von $f=100$~MHz hat eine Induktivität von 160~nH genau
$X_L=100$~$\Omega$. Dennoch wird die Spannung nicht halbiert, stattdessen wird eine Phasenverschiebung sichtbar.}
    \label{fig:phasenverschiebung}
\end{figure}

\subsection{Das Smith-Diagramm}
Aus dem letzten Abschnitt wird klar: Die betrachteten Bauteile Widerstand, Kondensator und Spule haben Wirk- und
Blindwiderstände, und eine Serienschaltung aus einem Widerstand von
$R=100$~$\Omega$ und einer Spule mit einem Blindwiderstand von $X_L=100$~$\Omega$ hat nicht einen Ersatzwiderstand
von 100~$\Omega$. Bis auf weiteres werden Wirk- und Blindwiderstand als separate Grössen geführt. Ein Bauteil,
oder eine Kombination von Bauteilen, kann auf diese Art einen Wirkwiderstand
$R\ge0$, wie auch einen Blindwiderstand $X$ haben. Ist der Blindwiderstand positiv, so handelt es sich um ein
Induktivität; ein negativer Blindwiderstand beschreibt eine Kapazität. Für ein Bauteil oder eine Schaltung
aus Bauteilen kann also z.B. gelten: $R=200$~$\Omega$, $X=100$~$\Omega$. Diese Grösse wird Impedanz $Z$ genannt
und lässt sich in einem Koordinatensystem als Punkt darstellen, wenn $R$ auf
der horizontalen und $X$ auf der vertikalen Achse abgetragen wird. Das Beispiel
ist in Abbildung \ref{fig:euler} in  blau gezeichnet. 

Auch im Smith-Diagramm wird genau diese Information (also eine Impedanz bestehend aus Wirk- und Blindwiderstand)
ebenfalls als Punkt dargestellt. Ein Punkt im Smith-Diagramm unterscheidet sich in dieser Hinsicht also nicht
von einem Punkt im eingangs besprochenen Koordinaten-System. Der Unterschied ist lediglich, dass im Smith-Diagramm
die ``Achsen'' eine andere Form, welche auf den ersten Blick sehr sonderbar erscheinen mag. Punkt (und damit Impedanzen)
im Smith-Diagramm folgen diesen Regeln:


\begin{enumerate}
\item Üblicherweise wird das Smith-Diagramm auf 50~$\Omega$ normalisiert, so dass
    der Punkt genau in der Mitte $R=50$~\Ohm und $X=0$~\Ohm aufweist.
\item Auf der Horizontalen in der Mitte des Diagrammes sind alle Impedanzen mit $X=0$~\Ohm, also alle Impedanzen
    ohne Blindanteil.
\item Der Punkt ganz links auf dieser Horizontalen ist $R=0$~\Ohm, also eine Impedanz ohne Wirk- und Blindwiderstand
    bzw. ein Kurzschluss.
\item Der Punkt ganz rechts auf der selben Horizontalen ist $R=\infty$~\Ohm, also eine offene Verbindung.
\item Punkte oberhalb der Horizontalen sind induktive Impedanzen ($X>0$~\Ohm).
\item Punkte unterhalb der Horizontalen sind kapazitive Impedanzen ($X<0$~\Ohm).
\item Punkte auf der umgebenden Kreislinie haben keinen Wirkwiderstand, sie sind also rein induktiv bzw. kapazitiv.
\end{enumerate}

\begin{figure}[t]
    \begin{center}
        \begin{tikzpicture}
            \begin{axis}[
                xmin=-5, xmax=355,
                ymin=-155,ymax=155,
                xlabel={$R/\Omega$},
                ylabel={$X/\Omega$},
                grid=both,
                height=7cm,
                axis x line=center,
                axis y line=center,
                width=0.9\textwidth,
                major grid style={black!10}
            ]
            \fill [fill=HB9UFblue] (axis cs:200,100) circle [radius=3pt] node[right] {$R=200$~\Ohm, $X=100$\Ohm};
            \fill [fill=black] (axis cs:50,0) circle [radius=2pt] node[above right] {$R=50$~\Ohm, $X=0$~\Ohm};
            \fill [fill=black] (axis cs:0,100) circle [radius=2pt] node[right] {$R=0$~\Ohm, $X=100$~\Ohm (Spule)};
            \fill [fill=black] (axis cs:0,-100) circle [radius=2pt] node[right] {$R=0$~\Ohm, $X=-100$~\Ohm (Kondensator)};
            \fill [fill=black] (axis cs:0,0) circle [radius=2pt];
            \draw[stealth'-] (axis cs:5,5) -- (axis cs:40,40) node[above right] {$R=0$~\Ohm, $X=0$~\Ohm (Kurzschluss)};
        \end{axis}
        \end{tikzpicture}
    \end{center}
    \caption{Darstellung einer Impedanz mit $R=200$~$\Omega$ auf der horizontalen Achse und $X=100$~$\Omega$
    auf der vertikalen Achse (blau), sowie einer Reihe anderer Impedanzen (schwarz).}
    \label{fig:euler}
\end{figure}

In Abbildung \ref{fig:smithimpedance} sind die selben Impedanzen zu sehen wir in Abbildung \ref{fig:euler}, nur
sind sie nun in einem Smith-Diagramm dargestellt. Berechtigterweise stellt sich die Frage, wozu denn das Ganze
gut sein soll.

\begin{itemize}
    \item Richtung der Impedanzen, Matching
    \item Leitungstheoretische Aspekte
\end{itemize}

\begin{figure}
    \begin{tikzpicture}
        \begin{smithchart}[width=\textwidth,show origin code/.code={},clip=false,xticklabel style={font=\small},yticklabel style={font=\small}]
            \tikzstyle{note}=[align=center,fill=black]
            \fill [note,fill=HB9UFblue] (axis cs:4,2) circle [radius=4pt] node[above] {$R=200$~\Ohm, $X=100$\Ohm};
            \fill[note] (axis cs:1,0) circle [radius=3pt] node[below] {$R=50$~\Ohm, $X=0$~\Ohm};
            \fill[note] (axis cs:0,2) circle [radius=3pt] node[below left] {$R=0$~\Ohm, $X=100$~\Ohm\\ (Spule)};
            \fill[note] (axis cs:0,-2) circle [radius=3pt] node[above left] {$R=0$~\Ohm, $X=-100$~\Ohm\\ (Kondensator)};
            \fill[note] (axis cs:0,0) circle [radius=3pt] node[below right] {$R=0$~\Ohm, $X=0$~\Ohm\\ (Kurzschluss)};
            \fill[note] (axis cs:1000,1000) circle [radius=3pt] node[below left] {Offen};
        \end{smithchart}
    \end{tikzpicture}
    \caption{Smith-Diagramm der Impedanzen aus Abbildung \ref{fig:euler}.}
    \label{fig:smithimpedance}
\end{figure}

\subsection{Kalibrierung}

Zur Korrektur von Messabweichungen in Testkabeln und dem Gerät
(Frequenzabhängkeit von Stimulus und Detektor) muss vor jeder Messung eine
Kalibrierung durchgeführt werden. Wie in Abschnitt \ref{sec:tl} erläutert,
werden Impedanzen durch Leitungen transformiert. Eine Kalibrierung macht
auch diesen Effekt rückgängig. Dafür wird ein Kalibrierungs-Standard
benötigt; in Abbildung \ref{fig:calkit} sind zwei solche Standards abgedruckt.
Es gibt eine ganze Reihe von Kalibrierungstechniken, welche z.B.
in \cite{hiebel2007fundamentals}  und \cite{hiebel2008vector} erklärt sind.
An dieser Stelle werden für Reflexions- und Transmissionsmessung je eine
einfache Kalibrierung vorgestellt:

\subsubsection{Reflexionsmessung}
Diese Messung wir mit einer offenen Verbindung (open, O), einem Kurzschluss
(short, S) und einer 50~\Ohm Last (load bzw. match, L/M) durchgeführt. Diese Art von
Kalibrierung wird deshalb auch OSL- bzw. OSM-Kalibrierung genannt. Diese
werden der Reihe nach angeschlossen und das Messgerät über den aktuell angeschlossenen
Abschluss orientiert. Dabei ist es wichtig, dass der Standard dort angeschlossen
wird, wo auch die Messung erfolgen soll. Das Gerät kann so die Transformation
der Messleitung sowie weitere Fehler rückgängig machen.

\subsection{Transmissionsmessung}
Bei der Transmissionsmessung wird das Gerät mit einem Verbindungsstück (thru, T) kalibriert,
welcher dann entfernt bzw. durch den Prüfling ersetzt wird. So weiss das  Gerät,
wieviel Energie in welcher Phase vom einen in den anderen Port gelangt, wenn der
Prüfling nicht eingeschleift wird. Diese Messung wird dann als 0~dB bzw. 0° Referenz
genutzt. Jeder Fehler im Verbindungsstück (Dämpfung $>$ 0~dB, Phasengang $\ne$~0°)
wirkt sich direkt auf die Messung aus.


\section{Bedienung NanoVNA}
\begin{itemize}
    \item Frequenzeinstellung
    \item Kanaleinstellung
    \item Kalibrierung
    \item Trace-Optionen
\end{itemize}

\section{FIXME: Antenne}

\section{FIXME: Abschlüsse/Impedanzen: Widerstände, Spulen, Kondensatoren}
\begin{itemize}
    \item Vergleich mit Bauteiltester
    \item Parasitäre Kapazität in Spulen
    \item Koaxialkabel (insb. 75 Ohm)?
    \item Langes, schleches Koax?
    \item Dummy Load?
\end{itemize}

\newpage % FIXME
\section{Schwingkreise, Resonanzfrequenzen, Anpassung}\label{sec:schwingkreis}
% FIXME: Nach Bauteil-Impedanz
\subsection{Theorie}
Es gibt mehrere Gründe, sich mit der Messung von Schwingkreisen
auseinanderzusetzen. Der einfachste besteht darin, die Resonanzfrequenz
eines solchen zu ermitteln -- dies schliesst auch die Eigenresonanzfrequenz von
Bauteilen ein. Dann ist es möglich, Spulen und Kondensatoren genauer zu
bestimmen, indem sie zu einem Schwingkreis geschaltet werden: Ist die
Induktivität einer Spule gesucht, wird diese Spule mit einem Kondensator
mit bekannter Kapazität zu einem Schwingkreis verschaltet, die Resonanzfrequenz
gemessen und mit der gemessenen Resonanzfrequenz $f$ und der bekannten
Kapazität $C$ die gesuchte Induktivität $L$ berechnet. Schliesslich sind
Schwingkreise einfache Modelle von Antennen.

\subsubsection{Schwingkreis und Resonanzfrequenz}
Ein Schwinkreis ergibt sich, wenn ein Widerstand $R$, eine Induktivität $L$
und eine Kapazität $C$ in Serie geschaltet werden (Serienschwingkreis, siehe
Abbildung \ref{fig:schwingkreis}). Es besteht auch die Möglichkeit, diese drei
Bauteile parallel zu schalten (Parallelschwingkreis).

\begin{figure}[H]
    \begin{center}
    \begin{circuitikz} \draw
        (0,0) to[short,o-] (1,0) to[L=$L$] (3,0) to[C=$C$] (5,0) to[R=$R$] (7,0) to[short,-o] (8,0);
    \end{circuitikz}
    \end{center}
    \caption{Ein Serienschwingkreis bestehend aus einer Spule/Induktivität $L$,
    einem Kondensator/Kapazität $C$ und einem Widerstand $R$.}
    \label{fig:schwingkreis}
\end{figure}

Bei einer bestimmten Frequenz, der Resonanzfrequenz $f_r$, heben sich die
Blindwiderstände von $L$ und $C$ genau auf. Es gilt:

$$
f_r = \frac{1}{2\pi\sqrt{L\cdot C}}
$$

Sind also $L$ und $C$ bekannt, so lässt sich $f_r$ berechnen, ebenso lässt ich
$L$ berechnen, falls $C$ und $f_r$ bekannt sind etc. Abbildung \ref{fig:resonanzplot}
zeigt den Blindwiderstand (links, gestrichelte Linie) im Serienschwingkreis als Funktion der Frequenz. Es
handelt sich dabei um die Darstellung der Grösse $X_L-X_C$ aus Abbildung \ref{fig:XRLC}.
Unterhalb der Resonanzfrequenz ist der Blindwiderstand negativ, der
Schwingkreis hat also eine kapazitive Komponente bzw. der Kondensator
dominiert. Oberhalb der Resonanzfrequenz ist der Blindwiderstand positiv, der
Schwingkreis ist induktiv bzw. die Spule dominiert. Bei einer Resonanzfrequenz
von $f_r\approx$5~MHz ist der Blindwiderstand 0~$\Omega$, lediglich $R=50$~$\Omega$
ist am Eingang sichtbar. Ein Abschluss von 50~$\Omega$ entspricht einer perfekten
Anpassung und deshalb entsteht bei dieser Frequenz überhaupt kein Rückfluss.
Die selben Sachverhalte sind auf der rechten Seite von Abbildung
\ref{fig:resonanzplot} im Smith-Diagram erkennbar. Die Analogie zu Verhalten
einer Antenne im Umfeld ihrer Resonanzfrequenz ist offensichtlich.
\begin{figure}
\begin{center}\begin{tikzpicture}
\begin{axis}[
    xmin=1, xmax=9,
    ymin=-70,ymax=10,
    grid=both,
    xlabel={Frequenz $f$/MHz},
    ylabel={\uline{R\"uckfluss}},
    height=6cm,
    width=0.45\textwidth,
    major grid style={black!10}
]
\addplot[mark=none,color=HB9UFblue,thick]  table[x index = {0}, y index = {1}] {schwingkreis.dat};
\end{axis}
\begin{axis}[
    xmin=1, xmax=9,
    ymin=-280,ymax=40,
    grid=none,
    separate axis lines,
    ylabel style = {align=center},
    ylabel={\udash{Blindwiderstand $X/\Omega$}},
    ticks=none,
    xmajorticks=false,
    xminorticks=false,
    ymajorticks=true,
    yminorticks=false,
    scaled ticks=false,
    ytick = {-240,-160,-80,0},
    axis y line*=right,
    height=6cm,
    width=0.45\textwidth,
    major grid style={black!10}
]
\addplot[mark=none,dashed,color=HB9UFred,thick]  table[x index = {0}, y index = {2}] {schwingkreis.dat};
\fill [fill=black] (axis cs:5.033,0) circle [radius=2pt];
\node [below]  at  (axis cs:5.033,0) {$f_r$};
\end{axis}
\end{tikzpicture}
\hfill
\begin{tikzpicture}
    \begin{smithchart}[width=0.46\textwidth]
\pgfplotsset{yticklabel in circle}
\addplot[thick,smooth,color=HB9UFblue] table[x index = {3}, y index = {4}] {schwingkreis.dat};
\fill [fill=black] (axis cs:1,0) circle [radius=2pt];
\node [below left]  at  (axis cs:1,0) {$f_r$};
\end{smithchart}
\end{tikzpicture}\end{center}
\caption{Links: Rückfluss und Blindwiderstand des Schwingkreises aus Abbildung \ref{fig:schwingkreis} mit 
$L$=1~$\mu$H, $C$=1~nF, $R$=50~$\Omega$. Rechts: Smith-Diagramm der Eingangsimpedanz dieser Schaltung.}
\label{fig:resonanzplot}
\end{figure}

\subsubsection{Anpassung}
Manchmal ist es wünschenswert, eine Impedanz in eine andere umzuwandeln. Dies ist beispielsweise der Fall,
wenn ein hochomiges Bauteil in einem 50~$\Omega$ System verwendet werden soll, oder wenn eine Last
(z.B. Antenne) auf der Zielfrequenz eine Impedanz aufweist, die fernab der Systemimpedanz liegt und eine
genügende Leistungsübertragung nicht erbracht wird. In diesem Fall kommen Transformatoren zur Impedanzwandlung
(siehe \ref{sec:impedanztrafo}), speziell konfigurierte Leitungen 
% FIXME: siehe ...
oder Anpass-Netzwerke zum Einsatz. Letztere bestehen aus Kombinationen von Kapazitäten und Induktivitäten,
mit denen die Impedanzanpassung vorgenommen wird. Die einfachste Möglichkeit hierfür ist der in Abbildung
\ref{fig:lmatch} gezeigte L-Match.

\begin{figure}[H]
    \scalebox{0.8}{
    \begin{circuitikz} \draw
        (0.5,0) to[short,o-] (1,0) to[C] (3,0) to[short,-o] (3.5,0)
                             (3,0) to[L,*-] (3,-2) node[ground] {};
     \end{circuitikz}}
    \hfill
    \scalebox{0.8}{
    \begin{circuitikz} \draw
        (0.5,0) to[short,o-] (1,0) to[L] (3,0) to[short,-o] (3.5,0)
                             (3,0) to[C,*-] (3,-2) node[ground] {};
     \end{circuitikz}}
    \hfill
    \scalebox{0.8}{
    \begin{circuitikz} \draw
        (0.5,0) to[short,o-] (1,0) to[L] (3,0) to[short,-o] (3.5,0)
                             (3,0) to[L,*-] (3,-2) node[ground] {};
     \end{circuitikz}}
    \hfill
    \scalebox{0.8}{
    \begin{circuitikz} \draw
        (0.5,0) to[short,o-] (1,0) to[C] (3,0) to[short,-o] (3.5,0)
                             (3,0) to[C,*-] (3,-2) node[ground] {};
    \end{circuitikz}}

    \vspace{3em}
    % zweite Reihe startet hier
    \scalebox{0.8}{
    \begin{circuitikz} \draw
        (0.5,0) to[short,o-] (1,0) to[C] (3,0) to[short,-o] (3.5,0)
                             (1,0) to[L,*-] (1,-2) node[ground] {};
     \end{circuitikz}}
    \hfill
    \scalebox{0.8}{
    \begin{circuitikz} \draw
        (0.5,0) to[short,o-] (1,0) to[L] (3,0) to[short,-o] (3.5,0)
                             (1,0) to[C,*-] (1,-2) node[ground] {};
     \end{circuitikz}}
    \hfill
    \scalebox{0.8}{
    \begin{circuitikz} \draw
        (0.5,0) to[short,o-] (1,0) to[L] (3,0) to[short,-o] (3.5,0)
                             (1,0) to[L,*-] (1,-2) node[ground] {};
     \end{circuitikz}}
    \hfill
    \scalebox{0.8}{
    \begin{circuitikz} \draw
        (0.5,0) to[short,o-] (1,0) to[C] (3,0) to[short,-o] (3.5,0)
                             (1,0) to[C,*-] (1,-2) node[ground] {};
    \end{circuitikz}}

    \caption{L-Match: Verschiedene Anordnungen von Kapazitäten und Induktivitäten zur Anpassung der Impedanz.}
    \label{fig:lmatch}
\end{figure}

Keine dieser Schaltungen ermöglicht die Anpassung einer beliebigen Impedanz zu
einer anderen beliebigen Impedanz. Es ist allerdings möglich, die Bauteilwerte
variabel zu gestalten und mehrere L-Matches zusammenzuschalten, um die
Anpassungemöglichkeiten zu erweiterten. Ein in diesem Sinn beliebter Ansatz 
ist der in Abbildung \ref{fig:tmatch} dargestellte T-Match.

\begin{figure}[H]
    \begin{center}
    \begin{circuitikz} \draw
        (0,0) to[short,o-] (1,0) to[vC,mirror] (3,0) to[vC,mirror] (5,0) to[short,-o] (6,0)
        (3,0) to[short,*-] (3,-1) to[vL,invert,mirror] (3,-3) node[ground] {};
    \end{circuitikz}
    \end{center}
    \caption{T-Match: Beliebte Schaltung zur Impedanzanpassung (Matchbox/Antennen-Tuner). Es handelt sich um einen
    Verbund zweier variabler L-Matches aus Abbildung \ref{fig:lmatch}.}
    \label{fig:tmatch}
\end{figure}

\subsection{Messung}
% Requirements_
% - RLC Schwingkreis
% - Alligator lead
% - Mess-Apparatur Reflexionsmessung
% - Tuner
% - Adapter for tuner
% - Extra Calkit


% FIXME: Zwei verschiedene Schwingkreise
Diese Messung ist zweiteilig. Zunächst wird ein Serienschwinkgkreis ($R$=50~$\Omega$a
$L$=FIXME~H und $C$=FIXME~F), als solches gemessen. Anschliessend wird er als
Antennenmodell verwendet, dessen Impedanz mit einem Tuner angepasst wird.

\begin{enumerate}
    \item Berechnen Sie die Resonanzfrequenz des Schwingkreises und stellen Sie den
        Frequenzbereich des NanoVNA auf einen sinnvollen Bereich ein, so dass die
        Resonanzfrequenz in der Mitte ist.
    \item Kalibrieren Sie den NanoVNA für eine Reflexionsmessung und schliessen Sie
        anschliessend den Schwingkreis (FIXME: mit welchem Adapter?) an.
    \item Betrachten Sie die Rückflussdämpfung (``LogMag''), wie auch Wirkwiderstand
        (``Resistance'') und Blindwiderstand (``Reactance'').
        \textbf{Entspricht das Resultat Ihren Erwartungen?}
    \item Betrachten Sie anschliessend die Messung im Smith-Diagramm. \textbf{Können Sie die
        Anzeige erklären?}
    \item Schliessen Sie den Widerstand für diese Teilaufgabe mit einer Klemme
        kurz. \textbf{Wie verändert sich die Rückflussdämpfung und die Anzeige
        im Smith-Diagramm?}
    \item Fügen Sie eines Ihrer Koaxial-Messkabel zwischen dem Schwingkreis und dem
        NanoVNA ein. \textbf{Wie verändert sich die Anzeige im Smith-Diagramm? In welchem
        Zusammenhang steht diese Relation mit der Länge des Messkabels?}
        %\hline FIXME
    \item Betrachten Sie das Innere des Tuners und identifizieren Sie die Bauteile aus 
        Abbildung \ref{fig:tmatch}.
    \item Schleifen Sie den Tuner zwischen Modell-Antenne und Messgerät und erneuern Sie die
        Kalibrierung? \textbf{Wo schliessen Sie das Kalibirierungskit an, um die Impedanz
        am Eingang des Tuners zu sehen?}
    \item Schalten Sie den Tuner auf ``Bypass'' und stellen Sie sicher, dass Sie das selbe
        Messresultat sehen wie bisher.
    \item Verwenden Sie den Tuner, um unter Betrachtung der Rückführungsdämpfung eine
        Anpassung bei FIXME~MHz zu erzielen. \textbf{Was geschieht beim Betätigen der
        Abstimmknöpfe mit dem Dämpfungsminimum?}
    \item Reduzieren Sie die Frequenzspanne so, dass im Smith-Diagramm lediglich ein
        kleiner Punkt für FIXME~MHz angezeigt wird. Wiederholen Sie die
        vorausgehende Teilaufgabe und \textbf{vergewissern Sie sich, dass die Anpassung
        dann erfolgt, wenn der Punkt ins Zentrum des Diagrammes wandert}.
    \item Machen Sie die Reduktion der Frequenzspanne rückgängig und betrachten Sie, wie
        der Tuner die ganze Impedanzkurve im Smith-Diagramm beeinflusst.
\end{enumerate}

\section{Dämpfungsglieder, Koppler und Isolation}
\subsection{Theorie}
Die Dämpfung oder Isolation zwischen zwei Anschlüssen von Dämpfungsgliedern,
Stufendämpfungsgliedern, Splitter-Combinern, Richt-, Hybrid- und anderer Koppler
kann mit einer Transmissionsmessung ermittelt werden. Dazu wird der VNA auf den
gewünschten Frequenzbereich eingestellt, mit einem Verbindungsstück kalibriert
und dann eingeschleift. Auf die selbe Art und Weise können Filter (Abschnitt 
\ref{sec:filter}) und Stub-Filter (Abschnitt \ref{sec:stubfilter}) charakterisiert
werden.

Bei vollständiger Isolation wäre die Einfügedämpfung unendlich gross. Trotzdem
zeigt ein VNA einen endlichen Wert an. Es kann keine Dämpfung/Isolation gemessen
werden, welche diesen Wert übersteigt.

\subsection{Messung}
% Requirements:
% 30 dB att, 3 &  6 dB att with adapters
% 3 Step attenuators with adapters
% 2 directional couplers (homebrew and HP) with adapters with adapters
% FIXME: Hybrid coupler?
% FIXME: Antennenschalter?
% FIXME: Einspeiseweiche?
% Wilkinson
Es stehen verschiedene Geräet zur Charakterisierung bereit: Fixe Dämpfungsglieder,
Stufendämpfungsglieder, Richtkoppler, Hybridkoppler und ein Wilkinson-Splitter.
Starten Sie bei jeder Messung mit dem vollen Messbereich.

\begin{enumerate}
    \item Ermitteln Sie nach der Kalibrierung die Einfügedämpfung bei
        perfekter Isolation. Damit wissen Sie, welche Dämpfung Sie maximal
        messen können.
    \item \textbf{Dämpfungsglieder:} Vermessen Sie die zu Verfügung stehenden
        Dämpfungsglieder (3~dB, 6~dB, 10~dB) und nach Belieben Kombinationen davon.
    \item \textbf{Stufendämpfungsglieder:} Vergleichen Sie das Stufendämpfungsglied
        für Kurzwelle von MFJ mit einem professionellen Stufendämpfungsglied.
        \textbf{In welchem Frequenzbereich können diese Geräte eingesetzt werden?}
    \item \textbf{Richtkoppler:} Auch hier stehen zwei Richtkoppler zur charakterisierung
        zu Verfügung. % FIXME
    \item \textbf{FIXME: Hybrid-Koppler}
    \item \textbf{Splitter-Combinder:} Gemäss Aliexpress Webseite ist dieser
        Splitter-Combiner für das 70~cm Band ausgelegt. \textbf{Überprüfen Sie
            diese Behauptung und charakterisieren Sie die Dämpfung (ein idealer
            Wilkinson-Splitter hat eine Dämpfung von 3~dB) sowie Isolation mit
            und ohne Abschluss.}
\end{enumerate}


\section{Filter}\label{sec:filter}

\subsection{Theorie}
Filter lassen einige Frequenzen des Spektrums passieren (Passband), während
sie andere Frequenzen reflektieren oder dissipieren. Die Pass- und Sperr-Charakteristika von Filtern werden üblicherweise mit einer
Transmissionsmessung bestimmt. Unter Umständen ist allerdings
auch eine Reflexionsmessung zielführend, z.B. wenn ein Duplexer abgestimmt
werden muss. Handelt es sich bei dem Filter nicht um ein 50~$\Omega$ System
(z.B. ein Keramik-Filter), muss die Fehlanpassung berücksichtigt werden.

\subsection{Messung}
Es werden eine Reihe von Filter (FIXME Bandpass-Filter für das 40~m Band,
diverse mechanische Kavitäten und Duplexer) auf ihre Eigenschaften hin
untersucht.

\begin{enumerate}
    \item Wählen Sie einen Filter für Ihre Messung aus. Falls Sie eine 70~cm
        Duplexer, Tiefpass oder Hochpass wählen, bietet es sich an, das Experiment
        mit einem Filter mit mehr bzw. weniger Kavitäten zu Wiederholen und
        allfällige Unterschiede zu begutachten.
    \item Stellen Sie die gewünschte Zielfrequenz ein und kalibrieren Sie das
        Gerät für eine Transmissionsmessung. Dabei ist darauf zu achten, dass
        bei Zuhilfenahme oder Weglassen von Adapter zwischen Kalibrierung und
        Messung, ein Fehler eingeführt wird.
    \item Schleifen Sie das Filter ein und charakterisieren Sie es.
    \item Konsultieren Sie im aktuellen Aufbau auch die Reflexionsmessung.
    \item Zusatzfrage zur Messung der Kavitäten: Ein UHF Umsetzer benötigt eine
        Isolation von ca. 80~dB. Kann dieser Umstand mit dem NanoVNA überprüft werden?


\end{enumerate}

\section{Koaxiale Stub-Filter}\label{sec:stubfilter}
\subsection{Theorie}
Eine Leitung -- z.B. ein Koaxialkabel -- der Länge $\lambda/4$ hat besondere
Eigenschaften: Ein Kurzschluss am einen Ende sieht vom anderen Ende aus wie
eine offene Verbindung; eine offene Verbindung am einen Ende sieht vom anderen
Ende aus wie ein Kurzschluss. Die Impedanz wird also transformiert, die
Struktur wird deshalb auch Viertelwellentransformator genannt. Im Smith-Diagramm
wird eine Abschlussimpedanz am Ende der Leitung in einem Halbkreis im Gegenuhrzeigersinn
rotiert. Zwei $\lambda/4$-Leitungen hintereinander entsprechen einer Leitung
der Länge $\lambda/2$, im Smith-Diagramm erfolgen zwei Rotationen um je einen
Halbkreis bzw. eine Rotation um einen Vollkreis: Die $\lambda/2$-Leitung
transformiert eine beliebige Impedanz (abgesehen von Verlusten im Kabel) auf
sich selbst.

Diese Eigenschaft kann genutzt werden, um mit Koaxialkabel einen relativ
schmalbandigen Sperr-Filter (Notch) zu implementieren. Dazu wird, wie in Abbildung
\ref{fig:quarterwavestub} gezeigt, mit einem T-Stück die $\lambda/4$-Leitung
parallel geschaltet. 
% FIXME: Alternative HB9UF print

Die Viertelwellentransformation mag etwas verwirrend erscheinen, ein Zahlenbeispiel
soll deshalb an dieser Stelle für etwas Klarheit sorgen: Für eine Frequenz von
30~MHz ist die Wellenlänge ca. $\lambda$=10~m. Eine Halbwellenleitung hat deshalb
die Länge $\lambda/2$=5~m und eine Vierteilwellenleitung $\lambda/4$=2.5~m. Ein
Kurzschluss am Ende dieser $\lambda/4$ Leitung wird deshalb bei 30~MHz zu einer
offenen Verbindung transformiert, welche die beiden anderen am T-Stück
angeschlossenen Leitungen keinen Einfluss hat. Bei der doppelten Frequenz, also
60~MHz, hat diese Leitung aber nicht mehr die Länge  $\lambda/4$, sondern
$\lambda/2$! Da die Leitung am Ende nach wie vor kurzgeschlossen ist, wird dieser
Kurzschluss bei 60~MHz also unmittelbar in das T-Stück transformiert, und ein
Kurzschluss führt zu einer Totalreflexion: Die Struktur sperrt bei 60~MHz. Das
selbe Phänomen tritt auch bei 120~MHz, 180~MHz etc. auf. Die Einfügedämpfung
ist in Abbildung \ref{fig:quarterwaveplot} dargestellt. Die gestrichelte Linie
zeigt ein Stub mit offenem (anstelle eines kurzgeschlossenen) Ende. Die Sperrwirkung
erfolgt dort bereits bei 30~MHz, da bei dieser Frequenz das offene Ende in einen
Kurzschluss transformiert wird.

\begin{figure}
\begin{center}
\begin{tikzpicture}
\begin{axis}[
    xmin=0, xmax=300,
    ymin=-35, ymax=5,
    grid=both,
    xlabel={Frequenz / MHz},
    ylabel={Transmission / dB},
    xtick = {0,30,...,300},
%   ticks=none,
    height=6cm,
    width=0.9\textwidth,
    major grid style={black!10}
]
\addplot[color=HB9UFblue,thick] table[,x index = {0}, y index = {1}] {quarterwave.dat};
\addplot[color=HB9UFred,thick,dashed] table[,x index = {0}, y index = {2}] {quarterwave.dat};
\end{axis}
\end{tikzpicture}
\end{center}
\caption{Einfügedämpfung eines Stub-Filters der Länge 2.5~m. Die ausgezogene Linie beschreibt den Filterverlauf
eines kurzgeschlossenen Stubs, wärend die gestrichelte Linie den Filterverlauf eines Stubs mit offenem Ende beschreibt.}
\label{fig:quarterwaveplot}
\end{figure}

Das Vorgehen für die Messung eines Stub-Filters mit dem NanoVNA unterscheidet
sich nicht von den übrigen Filtermessungen in Abschnitt \ref{sec:filter}. Weil
aber interessante leitungstheoretische Aspekte eine Rolle spielen, werden diese
Stub-Filter hier separat behandelt.

\subsection{Messung}
% Requirements:
% 2x BNC-SMA Adapter
% BNC T-Stück
% Versch. BNC Kabel
% BNC Female-Female Adapter
% Extra BNC-SMA Adapter für Abschluss

Es werden Stub-Filter verschiedenere Länge mit offenem und kurgeschlossenem Ende charakterisiert.

\begin{enumerate}
    \item Bauen Sie das Stub-Filter wie in Abbildung \ref{fig:quarterwavestub} auf und schliessen Sie ihn für eine
        Transmissionsmessung an den NanoVNA an.
    \item Kalibrieren Sie das Messgerät, während der Stub vom T-Stück getrennt ist.
    \item Schliessen Sie Leitungen verschiedener Länge an den dritten Anschluss des T-Stücks, nachdem Sie deren
        Länge gemessen haben. Sie können auch mehrere Kabel mit einem Adapter verbinden. Stimmen die Ergebnisse qualitativ?
    \item Schliessen Sie den Stub mit ihrem Kalibrierungs-Kit und einem Adapter kurz. Wie verändert sich die Filterkurve?
    \item Bestimmen Sie den Verkürzungsfaktor des verwendeten Koaxialkabels.
    \item Konsultieren Sie im aktuellen Aufbau auch die Reflexionsmessung.
\end{enumerate}


\section{Mantelwellensperren}
% FIXME: Kerne usw.

\subsection{Theorie}

Eine Mantelwelle ist eine Welle, die sich auf dem Aussenleiter (Mantel) eines
Koaxialkabels ausbreitet. Dies ist in aller Regel aus verschiedenen Gründen
unterwünscht; um die Mantelwelle zu unterbinden, wird eine Mantelwellensperre
eingesetzt. Diese Mantelwellensperre soll für Gegentaktsignale wie das
Empfangssignal von der Antenne zum Empfänger und das Sendesignal vom Sender
zur Antenne durchlässig sein; für Gleichtaktsignale wie die Mantelwelle soll
die Mantelwellensperre nicht durchlässig sein.

Die Messung einer Mantewellensperre hat deshalb drei Aspekte:

\begin{enumerate}
    \item Durchlässigkeit von Gegentaktsignalen.
    \item Dämpfung von Gleichtaktsignalen.
    \item Verhalten unter Last (Erwärmung etc.)
\end{enumerate}

Mit einem VNA werden typischerweise nur die ersten beiden Aspekte erfasst,
Erwärmung unter Last kann z.B. mit einem Thermomenter, einer Wärmebildkamera
etc. erfasst werden und wird an dieser Stelle nicht besprochen.

\subsubsection{Gegentaktmessung}\label{sec:gegentaktmessung}

Die Gegentaktmessung ist einfach: Der VNA wird für eine Transmissionsmessung
kalibriert, die Sperre wird eingeschleift und die Einfüngungsdämpdung wird
gemessen. Eine gute Mantelwellensperre sollte bei dieser Messung eine geringe
Einfügedämpfung aufweisen.

\subsubsection{Gleichtaktmessung}

Für die Gleichtaktmessung wird die Mantelwellensperre auf eine besondere
Art und Weise mit dem VNA verbunden: Der Eingang wird mit dem Stimulus,
der Ausgang mit dem detektierenden Port verbunden. Allerdings werden die
beiden beiden Leiter der Leitung jeweils kurzgeschlossen und mit dem
Innenleiter der Ports verbunden; die beiden Aussenleiter der Ports werden
kurzgeschlossen. Wenn die so präparierte Mantewellensperre als diskretes
Bauteil betrachtet wird (siehe FIXME), so entspricht diese Messung der
seriellen Transmissionsmessung dieses Bauteils. Eine gute Mantelwellensperre
sollte bei dieser Messung eine hohe Einfügedämpfung haben.

Für diese Messung steht eine spezielle Apparatur zu Verfügung, welche die
Aussenleiter der beiden VNA-Ports kurzschliesst. Die Mantelwellensperre
kann dann über  Klemmen mit den beiden Innenleitern verbunden werden.

\subsection{Messung}
Es werden eine Reihe von Mantelwellensperren unterschiedlicher Ausführung
(FIXME) und Qualität auf ihre Eigenschaften hin untersucht. 

\begin{enumerate}
    \item Wählen Sie eine Mantelwellensperre für Ihre Messung aus.
    \item Bereiten Sie den VNA für eine Transmissionsmessung im Kurzwellenbereich vor und kalibrieren Sie das Gerät.
    \item Prüfen Sie die Gegentakteigenschaften der Sperre wie in Abschnitt \ref{sec:gegentaktmessung} beschrieben.
    \item Entfernen Sie die Messkabel und befestigen Sie die Testapparatur für die Messung der Gleichtakteigenschaften (Siehe Abbildung \ref{fig:gleichtaktmessung}). %FIXME Bild
    \item Bewegen Sie die Klemmen und betrachten Sie die Auswirkungen auf die gemessene Dämpfung. Werden die beiden Klemmen nahe beieinander gehalten, nimmt die Dämpfung aufgrund von Kopplung ab.
    \item Verbinden Sie die beiden Klemmen und kalibrieren Sie das Gerät erneut für eine Transmissionsmessung.
    \item Schlissen Sie die Mantelwellensperre an, indem Sie an Ein- und Ausgang der Mantelwellensperre jeweils eine Klemme befestigen. Die Klemme soll jeweils beide Leitungen kurzschliessen oder mit dem Aussenleiter verbunden sein.
\end{enumerate}

\section{Schwingquarz}

\subsection{Theorie}
Ein Schwingquarz ist ein Bauteil, welches zur Erzeugung von Schwingungen oder
als Filter verwendet werden kann. Ein vereinfachtes Ersatzschaltbild ist in
Abbildung \ref{fig:quarzersatz} dargestellt. 

\begin{figure}[H]
    \begin{center}
    \begin{circuitikz} \draw
        (0,0) to[short,o-*] (1,0) to[short] 
            (1,-1) to[L=$L_1$] (3,-1) to[C=$C_1$] (5,-1) to[R=$R_1$] (7,-1) to[short] (7,0) to[short,*-o] (8,0)
        (1,0) to[short] (1,1) to[C=C$_0$] (7,1) to[short] (7,0);
    \end{circuitikz}
    \end{center}
    \caption{Vereinfachtes Ersatzschaltbild eines Schwingquarzes.}
    \label{fig:quarzersatz}
\end{figure}

Im Ersatzschaltbild ist zu erkennen, dass der Schwingquarz aus einem Serien-Schwingkreis ($L_1$, $C_1$, $R_1$)
und einer dazu parallelen Kapazität C$_0$ besteht. Der Quarz hat aufgrund dieser Schaltung zwei Resonanzstellen:
Wie in Abbildung \ref{fig:quarzidealplot} gezeigt erfolgt die erste beim
Nulldurchgang des Blindwiderstandes, bei der zweiten ist der Blindwiderstand
unendlich gross. Diese beiden Resonanzstellen werden Resonanzfrequenz $f_r$
bzw. Antiresonanzfrequenz $f_a$ genannt. $f_r$ ist stets etwas geringer als
$f_a$. Für die Frequenz des Schwingquarzes wird $f_r$ angegeben, welche der
Resonanzfrequenz des Serienschwingkreises aus $L_1$, $C_1$ und $R_1$
entspricht:

$$
f_r = \frac{1}{2\pi\sqrt{L_1\cdot C_1}}
$$



\begin{figure}
\begin{center}
\begin{tikzpicture}
\pgfplotsset{set layers}
\begin{axis}[
    xmin=15.89, xmax=15.95,
    ymin=-65, ymax=5,
    grid=both,
    ylabel={\uline{Transmission}},
    ytick = {-60,-50,...,0},
    ticks=none,
    height=6cm,
    width=0.9\textwidth,
    major grid style={black!10}
]
\addplot[color=HB9UFblue,thick] table[,x index = {0}, y index = {1}] {quarz.dat};
\end{axis}
\begin{axis}[
    xmin=15.89, xmax=15.95,
    ymin=-10000, ymax=10000,
    grid=none,
    separate axis lines,
    ylabel style = {align=center},
    xlabel={Frequenz},
    ylabel={\udash{Blindwiderstand $X/\Omega$}},
    ticks=none,
    ymajorticks=true,
    yminorticks=true,
    scaled ticks=false,
    ytick = {0},
    axis y line*=right,
    height=6cm,
    width=0.9\textwidth,
    major grid style={black!10}
]
\addplot[color=HB9UFred,thick,dashed,restrict x to domain=15.89:15.9234] table[,x index = {0}, y index = {2}] {quarz.dat};
\addplot[color=HB9UFred,thick,dashed,restrict x to domain=15.9236:15.95] table[,x index = {0}, y index = {2}] {quarz.dat};
\end{axis}
\end{tikzpicture}
\end{center}
\caption{Einfügedämpfung (ausgezogen) und Blindwiderstand $X$ (gestrichelt, negativ wenn kapazitiv) der Ersatzschaltung
aus Abbildung \ref{fig:quarzersatz}. Es sind zwei Resonanzstellen sichtbar.}
\label{fig:quarzidealplot}
\end{figure}

Mit einer Transmissionsmessung lassen sich $f_r$ und $f_a$ messen, wobei besonders $f_a$ von parasitären Komponenten
stark beeinflusst werden kann. Der Quarz wird für die Messung in die Zusatzapparatur eingespannt, die auch für die
Charakterisierung der Mantelwellensperre benutzt worden ist. Dabei müssen folgende Punkte beachtet werden:

\begin{itemize}
    \item Weil Schwingquarze sehr schmalbandig sind und der NanoVNA nur 101 Punkte aufnimmt, muss zur Verhinderung grosser Messfehler ein entsprechend kleiner Frequenzbereich ausgewählt werden.
    \item Weil das Ersatzschaltbild aus Abbildung \ref{fig:quarzersatz} lediglich eine Vereinfachung darstellt, werden bei der Messung in der Praxis oberhalb von $f_r$ Nebenresonanzen auftreten.
\end{itemize}

Es gibt weitere, teilweise bessere Methoden, einen Schwingquarz zu charakterisieren (mit einem VNA und auch mit
anderen Gerätschaften.) Für eine gute Übersicht sei auf \cite{kortke2002} verwiesen.

\subsection{Messung}

Es wird ein 16~MHz Schwingquarz charakterisiert.

\begin{enumerate}
    \item Befestigen Sie die Testapparatur für das Einschleifen von Bauteilen (FIXME) an Ihren NanoVNA.
    \item Decken Sie mit dem Frequenzbereich den gesamten Kurzwellenbereich ab und kalibrieren Sie den VNA für eine
        Durchgangsmessung, nachdem Sie die beiden Klemmen kurzgeschlossen haben.
    \item Montieren Sie den Quarz und führen Sie eine Transmissionsmessung durch? Finden Sie die Resonanzfrequenz und
        mit dem Marker und notieren Sie sie.
    \item Wählen Sie eine feinere Spanne von 15.75~MHz bis 16.25~MHz,
        rekalibrieren Sie das Gerät und suchen Sie die Resonanzfrequenz erneut. Wie stark unterscheiden sich die ermittelten Werte?
    \item Stellen Sie sicher, dass Sie den Grund für die Abweichung der beiden Messungen genau verstehen. Wie genau ist die zweite Messung?
    \item Beachten Sie die Nebenresonanzen, die in Abbildung \ref{fig:quarzidealplot} nicht auftreten.
    \item Wählen Sie eine Frequenzspanne, mit der fast nur noch die beiden Resonanzstellen zu sehen sind. Erhöhen Sie $C_0$, indem Sie einen 10~pF Kondensator parallel zum Schwingquarz schalten. Wie verändern sich die Resonanzstellen, und was bedeutet dies im Zusammenhang mit der externen Beschaltung des Quarzes?
    \item Untersuchen Sie das Oberschwingungsverhalten des Bauteils.

\end{enumerate}

\section{FIXME: Transformatoren zur Impedanzwandlung}\label{sec:impedanztrafo}

\section{FIXME (zu erwägen): Antennengewinn}

\section{FIXME (zu erwägen): TDR}

\section{FIXME (zu erwägen): Software NanoVNA Saver}

\section{FIXME (zu erwägen): Firmware Upgrade}

\section{FIXME (zu erwägen): Zubehör}

\section{FIXME (zu erwägen): Funktionsprinzip}

%   https://www.nonstopsystems.com/radio/frank_radio_coax-sw.htm

%   Grid dip meter

%   Zubehör:
%   - USB C Kabel (gute Qualität)
%   - USB C Adapter
%   - Anderes Calkit
%   - Stift
%   - Schutzfolie
%   - Case
%   - Testprint
%   - Ladestrom
%   - Firmware mods
%https://www.rtl-sdr.com/nanovna-version-2-0-first-pcb-pictures-released-nanovna-naming-credit-clarifications/
%http://ormpoa.altervista.org/firmware.htm
% Weiterführende Literatur

\section{Menü-Übersicht} % FIXME Is this printed?

\newpage % FIXME

\scalebox{0.93}{\rotatebox{90}{\begin{tikzpicture}[font=\sffamily]
    \definecolor{marker}{HTML}{f9f871};
    \definecolor{stimulus}{HTML}{ffc75f};
    \definecolor{cal}{HTML}{ff9671};
    \definecolor{recall}{HTML}{ff6f91};
    \definecolor{config}{HTML}{d65db1};
    % Spare color: #845ec2
    \definecolor{trace0}{RGB}{255,200,0};
    \definecolor{trace1}{RGB}{0,155,255};
    \definecolor{trace2}{RGB}{0,220,0};
    \definecolor{trace3}{RGB}{220,0,220};
    \tikzstyle{menu}=[draw, text width=5em, text centered, minimum height=2em, node distance=2em,on grid]
    \tikzstyle{marker}=[fill=marker]
    \tikzstyle{stimulus}=[fill=stimulus]
    \tikzstyle{cal}=[fill=cal]
    \tikzstyle{recall}=[fill=recall]
    \tikzstyle{config}=[fill=config]
    % Main
    \node[menu] (display) {Display};
    \node[menu,marker,below of=display] (marker) {Marker};
    \node[menu,stimulus,below of=marker] (stimulus) {Stimulus};
    \node[menu,cal,below of=stimulus] (cal) {Cal};
    \node[menu,recall,below of=cal] (recall) {Recall};
    \node[menu,config,below of=recall] (config) {Config};

    % Display
    \node[menu,right=30em of display] (trace) {Trace};
    \node[menu,below of=trace] (format) {Format};
    \node[menu,below of=format] (scale) {Scale};
    \node[menu,below of=scale] (channel) {Channel};
    \node[menu,below of=channel] (transform) {Transform};

    \node[menu,right=30em of trace] (trace0) {\colorbox{trace0}{Trace 0}};
    \node[menu,below of=trace0] (trace1) {\colorbox{trace1}{Trace 1}};
        \node[menu,below of=trace1] (trace2) {\colorbox{trace2}{Trace 2}};
        \node[menu,below of=trace2] (trace3) {\colorbox{trace3}{Trace 3}};

    \node[menu,right=20em of format]  (logmag) {Logmag};
    \node[menu,below of=logmag]  (phase) {Phase};
    \node[menu,below of=phase]  (delay) {Delay};
    \node[menu,below of=delay]  (smith) {Smith};
    \node[menu,below of=smith]  (swr) {SWR};
    \node[menu,below of=swr]  (more) {More...};

    \node[menu,right=10em of more]  (polar) {Polar};
    \node[menu,below of=polar]  (linear) {Linear};
    \node[menu,below of=linear] (real) {Real};
    \node[menu,below of=real]  (imag) {Imag};
    \node[menu,below of=imag]  (resistance) {Resistance};
    \node[menu,below of=resistance] (reactance) {Reactance};

    \node[menu,below right=14em and 20em of scale]  (scalediv) {Scale/Div};
    \node[menu,below of=scalediv]  (referenceposition) {Ref. Pos};
    \node[menu,below of=referenceposition]  (electricaldelay) {Elec. Delay};

    \node[menu,right=10em of channel] (channel0) {CH 0};
    \node[menu,below of=channel0] (channel1) {CH 1};

    \node[menu,below right=5em and 10em of transform] (transformon) {Transf. on};
    \node[menu,below of=transformon] (lowpassimpulse) {LP impulse};
    \node[menu,below of=lowpassimpulse] (lowpassstep) {LP step};
    \node[menu,below of=lowpassstep] (bandpass) {Bandpass};
    \node[menu,below of=bandpass] (window) {Window};
    \node[menu,below of=window] (velocityfactor) {Vel. Factor};

    \node[menu,below right=7em and 10em of window] (minimum) {Minimum};
    \node[menu,below of=minimum] (normal) {Normal};
    \node[menu,below of=normal] (maximum) {Maximum};

    % Marker
    \node[menu,marker,below right=26em and 30em of marker] (selectmarker) {Select M.};
    \node[menu,marker,below of=selectmarker] (markerstart) {Start};
    \node[menu,marker,below of=markerstart] (markerstop) {Stop};
    \node[menu,marker,below of=markerstop] (markercenter) {Center};
    \node[menu,marker,below of=markercenter] (markerspan) {Span};

    \node[menu,marker,right=10em of selectmarker] (marker1) {Marker 1};
    \node[menu,marker,below of=marker1] (markeri) {...};
    \node[menu,marker,below of=markeri] (marker4) {Marker 4};
    \node[menu,marker,below of=marker4] (alloff) {All Off};

    % Stimulus
    \node[menu,stimulus,right=20em of stimulus] (start) {Start};
    \node[menu,stimulus,below of=start] (stop) {Stop};
    \node[menu,stimulus,below of=stop] (center) {Center};
    \node[menu,stimulus,below of=center] (span) {Span};
    \node[menu,stimulus,below of=span] (cwfreq) {CW Freq};
    \node[menu,stimulus,below of=cwfreq] (pausesweep) {Pause Swp.};

    % Cal
    \node[menu,cal,right=10em of cal] (calibrate) {Calibrate};
    \node[menu,cal,below of=calibrate] (calsave) {Save};
    \node[menu,cal,below of=calsave] (reset) {Reset};
    \node[menu,cal,below of=reset] (correction) {Correction};

    \node[menu,cal,below right=12em and 10em of calibrate] (open) {Open};
    \node[menu,cal,below of=open] (short) {Short};
    \node[menu,cal,below of=short] (load) {Load};
    \node[menu,cal,below of=load] (isoln) {Isoln};
    \node[menu,cal,below of=isoln] (thru) {Thru};
    \node[menu,cal,below of=thru] (done) {Done};

    \node[menu,cal,below right=24em and 10em of calsave] (save0) {Save 0};
    \node[menu,cal,below of=save0] (savei) {...};
    \node[menu,cal,below of=savei] (save4) {Save 4};

    % Recall
    \node[menu,recall,below right=10 em and 10em of recall] (recall0){Recall 0};
    \node[menu,recall,below of=recall0] (recalli){...};
    \node[menu,recall,below of=recalli] (recall4){Recall 4};

    % Config
    \node[menu,config,below right=20em and 10em of config] (touchcal) {Touch Cal};
    \node[menu,config,below of=touchcal] (touchtest) {Touch Test};
    \node[menu,config,below of=touchtest] (configsave) {Save};
    \node[menu,config,below of=configsave] (version) {DFU};

    \draw[thick,-stealth'] (display) -- (trace);
    \draw[thick,-stealth'] (marker) -- +(25em,0em) |- (selectmarker);
    \draw[thick,-stealth'] (stimulus) -- (start);
    \draw[thick,-stealth'] (cal) -- (calibrate);
    \draw[thick,-stealth'] (recall) -- +(5.5em,0em) |- (recall0);
    \draw[thick,-stealth'] (config) -- +(4.5em,0em) |- (touchcal);

    \draw[thick,-stealth'] (trace) -- (trace0);
    \draw[thick,-stealth'] (format) -- (logmag);
    \draw[thick,-stealth'] (scale) -- +(15em,0em) |- (scalediv);
    \draw[thick,-stealth'] (channel) -- (channel0);
    \draw[thick,-stealth'] (transform) -- +(5em,0em) |- (transformon);
    \draw[thick,-stealth'] (window) -- +(5em,0em) |- (minimum);

    \draw[thick,-stealth'] (more) -- (polar);

    \draw[thick,-stealth'] (selectmarker) -- (marker1);

    \draw[thick,-stealth'] (calibrate) -- +(5.5em,0em) |- (open);
    \draw[thick,-stealth'] (calsave) -- +(4.5em,0em) |-(save0);
\end{tikzpicture}}}

\bibliographystyle{unsrt} % FIXME is in TOC?
\bibliography{nanovna_workshop_skript}


\end{document}
