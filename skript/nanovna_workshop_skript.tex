\documentclass[twoside,a4paper,11pt,halfparskip,DIV=11,notitlepage]{scrartcl}
\usepackage{scrpage2}
\usepackage{amsmath}
\usepackage[pdfauthor={},pdftitle={},pdfstartview=FitH,pdfborder={0 0 0}]{hyperref}
\usepackage[ngerman]{babel}

\author{Mathias Weyland, HB9FRV}
\title{NanoVNA Workshop}
\date{UHF-Gruppe der USKA, Version 1.0}
\pagestyle{scrheadings}
\setkomafont{pagehead}{\sf}

\usepackage[svgnames]{xcolor}
\usepackage{tikz}
\usetikzlibrary{shadows}
\addtokomafont{descriptionlabel}{\normalfont}
\usepackage{tcolorbox}
\tcbuselibrary{skins,breakable}
\definecolor{myblue}{RGB}{40,96,139}

\tcbset{
mybox/.style={
  breakable,
  enhanced standard,
  boxrule=0.4pt,titlerule=-0.2pt,drop fuzzy shadow,
  width=\linewidth,
  title style={top color=myblue!30,bottom color=myblue!0.5},
  overlay unbroken and first={
    \path[fill=myblue]
    ([xshift=5pt,yshift=-\pgflinewidth]frame.north west) to[out=0,in=180] ([xshift=20pt,yshift=-5pt]title.south west) -- ([xshift=-20pt,yshift=-5pt]title.south east) to[out=0,in=180] ([xshift=-5pt,yshift=-\pgflinewidth]frame.north east) -- cycle;
  },
  fonttitle=\Large\bfseries\sffamily,
  fontupper=\sffamily,
  fontlower=\sffamily,
  before=\par\medskip\noindent,
  after=\par\medskip,
  center title,
  toptitle=3pt,
  top=11pt,topsep at break=-5pt,
  colback=white
}}

\newtcolorbox{lernziele}[1][\linewidth]{mybox,width=#1,title=Lernziele}

\newcounter{uebungscounter}
\newcommand{\uebung}[1]{
    \stepcounter{uebungscounter}
    \section{Übung \arabic{uebungscounter}: #1}
    \renewcommand{\labelenumi}{\arabic{uebungscounter}.\arabic{enumi}}
}
\newcommand{\loesungen}{
    \section*{Lösungen zu Übung \arabic{uebungscounter}}
}
%\setcounter{secnumdepth}{0} % sections are level 1

\sloppy

\begin{document}
\maketitle
\thispagestyle{empty}

\tableofcontents

\vfill

%\includegraphics[width=\textwidth]{titelbild.png} % FIXME

\vfill

\newpage

%   \uebung{Erste Schritte}
%   In dieser Übung trainieren Sie den Umgang mit dem GNU Radio Companion (GRC) und lernen erste Blöcke kennen.

%   \begin{lernziele}
%   \begin{itemize}
%   \item Sie können in GNU Radio Blöcke einfügen, löschen, verbinden und trennen.
%   \item Sie können einfache Wellenformen mit \hl{Signal Source} generieren.
%   \item Sie können Signale im Zeit- und Frequenzraum visualisieren.
%   \item Sie wissen, dass das Spektrum eigentlich auch negative Frequenzen aufweist.
%   \item Sie können Flows in GRC starten und stoppen.
%   \item Sie verstehen, wann der \hl{Throttle}-Block zum Einsatz kommt.
%   \end{itemize}
%   \end{lernziele}

\section{Grundlagen}

\begin{itemize}
    \item Transmissions- vs. Reflexionsmessung
    \item SWR, RL, IL
    \item Complex Loads
    \item Smith Diagramm
    \item Calibration
\end{itemize}

\section{Experiment: Abschlüsse im Smith Diagramm}

\section{Experiment: Antenne messen}

\section{Experiment: Filter}

\section{Experiment: Impedanzwandler}

\section{Experiment: Balun}

\section{Experiment: Koppler}

\section{DIY: Eigene Last bauen}

\end{document}
