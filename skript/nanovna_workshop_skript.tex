\documentclass[twoside,a4paper,11pt,halfparskip,DIV=11,notitlepage]{scrartcl}
\usepackage{scrpage2}
\usepackage{amsmath}
\usepackage[pdfauthor={},pdftitle={},pdfstartview=FitH,pdfborder={0 0 0}]{hyperref}
\usepackage[ngerman]{babel}

\author{Mathias Weyland, HB9FRV}
\title{NanoVNA Workshop}
\date{UHF-Gruppe der USKA, Version 1.0}
\pagestyle{scrheadings}
\setkomafont{pagehead}{\sf}

\usepackage[svgnames]{xcolor}
\usepackage{tikz}
\usetikzlibrary{shadows}
\addtokomafont{descriptionlabel}{\normalfont}
\usepackage{tcolorbox}
\tcbuselibrary{skins,breakable}
\definecolor{myblue}{RGB}{40,96,139}

\tcbset{
mybox/.style={
  breakable,
  enhanced standard,
  boxrule=0.4pt,titlerule=-0.2pt,drop fuzzy shadow,
  width=\linewidth,
  title style={top color=myblue!30,bottom color=myblue!0.5},
  overlay unbroken and first={
    \path[fill=myblue]
    ([xshift=5pt,yshift=-\pgflinewidth]frame.north west) to[out=0,in=180] ([xshift=20pt,yshift=-5pt]title.south west) -- ([xshift=-20pt,yshift=-5pt]title.south east) to[out=0,in=180] ([xshift=-5pt,yshift=-\pgflinewidth]frame.north east) -- cycle;
  },
  fonttitle=\Large\bfseries\sffamily,
  fontupper=\sffamily,
  fontlower=\sffamily,
  before=\par\medskip\noindent,
  after=\par\medskip,
  center title,
  toptitle=3pt,
  top=11pt,topsep at break=-5pt,
  colback=white
}}

\newtcolorbox{lernziele}[1][\linewidth]{mybox,width=#1,title=Lernziele}

\newcounter{uebungscounter}
\newcommand{\uebung}[1]{
    \stepcounter{uebungscounter}
    \section{Übung \arabic{uebungscounter}: #1}
    \renewcommand{\labelenumi}{\arabic{uebungscounter}.\arabic{enumi}}
}
\newcommand{\loesungen}{
    \section*{Lösungen zu Übung \arabic{uebungscounter}}
}
%\setcounter{secnumdepth}{0} % sections are level 1

\sloppy

\begin{document}
\maketitle
\thispagestyle{empty}

\tableofcontents

\vfill

%\includegraphics[width=\textwidth]{titelbild.png} % FIXME

\vfill

\newpage

%   \uebung{Erste Schritte}
%   In dieser Übung trainieren Sie den Umgang mit dem GNU Radio Companion (GRC) und lernen erste Blöcke kennen.

%   \begin{lernziele}
%   \begin{itemize}
%   \item Sie können in GNU Radio Blöcke einfügen, löschen, verbinden und trennen.
%   \item Sie können einfache Wellenformen mit \hl{Signal Source} generieren.
%   \item Sie können Signale im Zeit- und Frequenzraum visualisieren.
%   \item Sie wissen, dass das Spektrum eigentlich auch negative Frequenzen aufweist.
%   \item Sie können Flows in GRC starten und stoppen.
%   \item Sie verstehen, wann der \hl{Throttle}-Block zum Einsatz kommt.
%   \end{itemize}
%   \end{lernziele}

\section{Grundlagen}

\begin{itemize}
    \item Transmissions- vs. Reflexionsmessung
    \item SWR, RL, IL
    \item Complex Loads
    \item Smith Diagramm
    \item Calibration
\end{itemize}

\section{Experiment: Abschlüsse im Smith Diagramm}

\section{Experiment: Antenne messen}

\section{Mantelwellensperren}

\subsection{Einleitung}

Eine Mantelwelle ist eine Welle, die sich auf dem Aussenleiter (Mantel) eines
Koaxialkabels ausbreitet. Dies ist in aller Regel aus verschiedenen Gründen
unterwünscht; um die Mantelwelle zu unterbinden, wird eine Mantelwellensperre
eingesetzt. Diese Mantelwellensperre soll für Gegentaktsignale wie das
Empfangssignal von der Antenne zum Empfänger und das Sendesignal vom Sender
zur Antenne durchlässig sein; für Gleichtaktsignale wie die Mantelwelle soll
die Mantelwellensperre nicht durchlässig sein.

Die Messung einer Mantewellensperre hat deshalb drei Aspekte:

\begin{enumerate}
    \item Durchlässigkeit von Gegentaktsignalen.
    \item Dämpfung von Gleichtaktsignalen.
    \item Verhalten unter Last (Erwärmung etc.)
\end{enumerate}

Mit einem VNA werden typischerweise nur die ersten beiden Aspekte erfasst,
Erwärmung unter Last kann z.B. mit einem Thermomenter, einer Wärmebildkamera
etc. erfasst werden und wird an dieser Stelle nicht besprochen.

\subsection{Gegentaktmessung}\label{sec:gegentaktmessung}

Die Gegentaktmessung ist einfach: Der VNA wird für eine Transmissionsmessung
kalibriert, die Sperre wird eingeschleift und die Einfüngungsdämpdung wird
gemessen. Eine gute Mantelwellensperre sollte bei dieser Messung eine geringe
Einfügungsdämpfung aufweisen.

\subsection{Gleichtaktmessung}

Für die Gleichtaktmessung wird die Mantelwellensperre auf eine besondere
Art und Weise mit dem VNA verbunden: Der Eingang wird mit dem Stimulus,
der Ausgang mit dem detektierenden Port verbunden. Allerdings werden die
beiden beiden Leiter der Leitung jeweils kurzgeschlossen und mit dem
Innenleiter der Ports verbunden; die beiden Aussenleiter der Ports werden
kurzgeschlossen. Wenn die so präparierte Mantewellensperre als diskretes
Bauteil betrachtet wird (siehe FIXME), so entspricht diese Messung der
seriellen Transmissionsmessung dieses Bauteils. Eine gute Mantelwellensperre
sollte bei dieser Messung eine hohe Einfügungsdämpfung haben.

Für diese Messung steht eine spezielle Apparatur zu Verfügung, welche die
Aussenleiter der beiden VNA-Ports kurzschliesst. Die Mantelwellensperre
kann dann über  Klemmen mit den beiden Innenleitern verbunden werden.

\subsection{Experiment}
Es werden eine Reihe von Mantelwellensperren unterschiedlicher Ausführung
(FIXME) und Qualität auf ihre Eigenschaften hin untersucht. 

\begin{enumerate}
    \item Wählen Sie eine Mantelwellensperre für Ihre Messung aus.
    \item Bereiten Sie den VNA für eine Transmissionsmessung im Kurzwellenbereich vor und kalibrieren Sie das Gerät.
    \item Prüfen Sie die Gegentakteigenschaften der Sperre wie in Abschnitt \ref{sec:gegentaktmessung} beschrieben.
    \item Entfernen Sie die Messkabel und befestigen Sie die Testapparatur für die Messung der Gleichtakteigenschaften (Siehe Abbildung \ref{fig:gleichtaktmessung}). %FIXME Bild
    \item Bewegen Sie die Klemmen und betrachten Sie die Auswirkungen auf die gemessene Dämpfung. Werden die beiden Klemmen nahe beieinander gehalten, nimmt die Dämpfung aufgrund von Kopplung ab.
    \item Verbinden Sie die beiden Klemmen und kalibrieren Sie das Gerät erneut für eine Transmissionsmessung.
    \item Schlissen Sie die Mantelwellensperre an, indem Sie an Ein- und Ausgang der Mantelwellensperre jeweils eine Klemme befestigen. Die Klemme soll jeweils beide Leitungen kurzschliessen oder mit dem Aussenleiter verbunden sein.
\end{enumerate}


\section{Filter}

\subsection{Einleitung}
Die Pass- und Sperr-Charakteristika von Filtern werden üblicherweise mit einer
Transmissionsmessung bestimmt. Unter Umstände. Unter Umständen ist allerdings
auch eine Reflexionsmessung zielführend, z.B. wenn ein Duplexer abgestimmt
werden muss. Handelt es sich bei dem Filter nicht um ein 50~$\Omega$ System
(z.B. ein Keramik-Filter), muss die Fehlanpassung berücksichtigt werden.

\subsection{Experiment}
Es werden eine Reihe von Filter (FIXME Bandpass-Filter für das 40~m Band,
diverse mechanische Kavitäten und Duplexer) auf ihre Eigenschaften hin
untersucht.

\begin{enumerate}
    \item Wählen Sie einen Filter für Ihre Messung aus. Falls Sie eine 70~cm
        Duplexer, Tiefpass oder Hochpass wählen, bietet es sich an, das Experiment
        mit einem Filter mit mehr bzw. weniger Kavitäten zu Wiederholen und
        allfällige Unterschiede zu begutachten.
    \item Stellen Sie die gewünschte Zielfrequenz ein und kalibrieren Sie das
        Gerät für eine Transmissionsmessung. Dabei ist darauf zu achten, dass
        bei Zuhilfenahme oder Weglassen von Adapter zwischen Kalibrierung und
        Messung, ein Fehler eingeführt wird.
    \item Schleifen Sie das Filter ein und charakterisieren Sie es.
    \item Zusatzfrage zur Messung der Kavitäten: Ein UHF Umsetzer benötigt eine
        Iolation von ca. 80~dB. Kann dieser Umstand mit dem NanoVNA überprüft werden?


\end{enumerate}


\section{Experiment: Impedanzwandler}

\section{Experiment: Balun}

\section{Experiment: Koppler}

\section{DIY: Eigene Last bauen}

\end{document}
